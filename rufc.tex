\documentclass[a4paper, 12pt]{book}
\usepackage[T2A]{fontenc}
\usepackage[utf8]{inputenc}
\usepackage[english, russian]{babel}
\usepackage{fancyvrb}
\usepackage{upquote}
\usepackage{textcomp}
\usepackage[pdftex, dvips]{graphicx}

\graphicspath{{Images/Cover/}{Images/Chapter_02/}}
\usepackage{pdfpages}
\usepackage[pdfborder={0 0 0}, pdftex]{hyperref}
\usepackage{amsmath}
\usepackage{listings}
\usepackage{epigraph}
\usepackage{tikz}
%\usetikzlibrary{graphdrawing}

%\usepackage{pgfmath}
\usetikzlibrary{graphs}
%\usetikzlibrary{graphdrawing.force}
%\usegdlibrary{trees}

\usetikzlibrary{calc, positioning, decorations, decorations.pathmorphing, decorations.pathreplacing}

%\usepackage{cm-super}
\title{Функциональный C}
\date{03.01.1999}
\author{Питер Хартел (Pieter Hartel) Хенк Мюллер (Henk Muller) \\ Саутгемптонский университет Бристольский университет}

\begin{document}
\lstdefinestyle{customoz}{
  belowcaptionskip=1\baselineskip,
  breaklines=true,
%  frame=L,
  xleftmargin=\parindent,
  language=Oz,
  showstringspaces=false,
  basicstyle=\footnotesize\ttfamily,
  keywordstyle=\bfseries\color{green!40!black},
  commentstyle=\itshape\color{purple!40!black},
  identifierstyle=\color{blue},
  stringstyle=\color{orange},
}

\lstdefinestyle{customc}{
  belowcaptionskip=1\baselineskip,
  breaklines=true,
  frame=L,
  xleftmargin=\parindent,
  language=C,
  showstringspaces=false,
  basicstyle=\footnotesize\ttfamily,
  keywordstyle=\bfseries\color{green!40!black},
  commentstyle=\itshape\color{purple!40!black},
  identifierstyle=\color{blue},
  stringstyle=\color{orange},
}
\lstset{language=C, mathescape=true, style=customc}

\renewcommand {\contentsname} {Оглавление}
\renewcommand {\chaptername} {Глава}
%\includepdf{Cover.pdf}

{\let\newpage\relax\maketitle}
\maketitle
\newpage

Посвящается Марике \emph{Питер}

Посвящается моей семье и другим источникам вдохновения \emph{Хенк}

\tableofcontents
\listoffigures
\listoftables

\chapter*{Предисловие}
\markboth{\MakeUppercase{Предисловие}}{}
\addcontentsline{toc}{chapter}{Предисловие}

Отделы Компьютерных Наук многих университетов в качестве первого языка программирования обучают функциональному языку. Использование функционального языка с его высоким уровнем абстракции помогает подчеркнуть принципы программирования. Функциональное программирование --- это только одна из парадигм с которыми должен быть знаком студент. Также важны Императивное, Параллельное, Объектно-Ориентированное и Логическое программирование. В зависимости от решаемой задачи выбирается одна из парадигм как наиболее естественная парадигма для этой задачи.

Эта книга --- материал курса для обучения второй парадигме: \emph{императивному программированию}, используя C в роли языка программирования. В этой книге предполагается, что студент усвоил его первый курс функционального программирования на примере SML. Требованием книги является понимание принципов программирования: эта книга не нацелена на 'обучение решению задач' или 'программированию'. Эта книга нацелена на:

\begin{itemize}
\item{Знакомство читателя с \emph{императивным программированием} как другим способом реализации программ. Цель --- сохранить стиль программирования, то есть, программист думает функционально при реализации императивной программы.}

\item{Дать понимание \emph{различия между функциональным и императивным программированием}. Функциональное программирование --- это высокий уровень деятельности. Порядок вычислений и выделение хранилища является автоматическим. Императивное программирование, особенно в C, --- это низкий уровень деятельности, где программист контролирует и порядок вычислений и выделение хранилища. Это усложняет императивное программирование, но даёт императивному программисту возможности для оптимизации, не доступные для функционального программиста.}

\item{Познакомить читателя с \emph{синтаксисом и семантикой ISO-C}, особенно с мощью языка (в то же время подчёркивая что мощь может и убить). Мы посетим все тёмные переулки C, с \lstinline|void *| до арифметики указателей и присваиваний в выражениях. В отдельных случаях мы используем другие языки (такие как C++ и Pascal) для демонстрации концепций императивного программирования, не существующих в C. C был выбран из-за того, что он де факто стандарт в декларативном программировании и поскольку его низкоуровневая природа красиво контрастирует с SML. Те, кто хочет изучить, к примеру, Modula-2 или Ada-95 не должны столкнуться с большими трудностями.}

\item{Улучшить \emph{знания принципов программирования} и \emph{навыков решения задач}. Этому способствует использование трёх различных языков (математика, функциональный язык и императивный язык). Тот факт, что эти крайне различные языки имеют общие аспекты подтверждает существование принципов программирования и естественность их пользы.}

\item{Улучшить \emph{знания принципов абстракций}. Во всей книге мы призываем студента искать более абстрактные решения, например, изучая сигнатуру функции как абстракцию её цели используя как процедурные абстракции (в частности высокоуровневые функции) на ранней стадии, так и абстракции данных.}

\item{Провести студента от спецификации и математики до реализации и \emph{построения программного обеспечения}. В первых главах мы делаем акцент на написании корректных функций, когда мы достигнем успехов, то акцент будет постепенно смещён в трансформирование корректных функций в эффективные и многократно используемые функции. Чистый и ясный интерфейс имеет первостепенное значение, и только в крайних случаях он будет приноситься в жертву лучшей эффективности.}
\end{itemize}

Каждая задача в этой книге решается в три шага:

\begin{itemize}
\item{Создание спецификации задачи.}

\item{Нахождение соответствующего алгоритма, дающего решение, удовлетворяющее спецификации.}

\item{Реализация как можно более эффективного алгоритма. Акцент всей книги приходится на этот третий шаг.}
\end{itemize}

Язык математики используется для описания задачи. Сюда входят основы теории множеств и логики. Студент должен быть немного знаком с исчислением множеств, логикой предикатов и логикой высказываний. Этот материал преподают в большинстве университетов в течении первого курса дискретной математики или формальной логики.

Соответствующие алгоритмы даны в SML. SML свободно доступен для широкого числа платформ (персональные компьютеры, рабочие станции UNIX, Apple) поэтому он популярен в роли обучающего языка. Поскольку многие функциональные языки не сильно отличаются от SML, то в приложении дан краткий обзор SML для тех, кто знаком с другими популярными языками программирования, такими как Miranda, Haskell, Clean или Scheme.

Поскольку перед языком мы ставили цель реализовать решения в императивном стиле, то выбор был остановлен на C. Выбор в использовании C, а не C++ был труден. Оба языка популярны, и следовательно пригодны для использования. Мы выбрали C поскольку он более чисто определяет низкоуровневое программирование. Для иллюстрации рассмотрим механизм предоставляемый языком для вызова через ссылку. В C аргументы должны явно передаваться в виде указателя. Вызывающая сторона должна передать \emph{адрес}, а вызываемая сторона должна \emph{разыменовать} указатель. В контрасте этому находится механизм вызова через ссылку в C++ (и Pascal и Modula-2). Этот явный вызов через ссылку является дидактическим качеством, поскольку он явно описывает модель, стоящую за вызовом по ссылке и её опасности (в виде нежелательных псевдонимов).

Поскольку эта книга предназначена для использования на первом курсе, то количество требований, предъявляемых для студентов очень мало. Рассуждения о корректности программ требуют навыков доказательств, которые возможно ещё не выработались у студентов на данном этапе. Поэтому мы обозначили все упражнения с доказательствами специальной меткой. Упражнения требующие доказательств помечены астериском. Мы считаем что книгу можно использовать не выполняя ни единого доказательства. Однако, при втором чтении мы рекомендуем студентам выполнять доказательства. Ответы к одной трети упражнений даны в Приложении А.

Студент должен иметь понимание базовых принципов вычисления. Сюда входят арифметика второго уровня и принципы операций машины фон Неймана. Этому лучше всего соответствует курс компьютерных вычислений. Книга включает примеры из других областей информатики, включая базы данных, компьютерную графику, теорию языков программирования и компьютерную архитектуру. Эти примеры можно понять не обладая какими-либо предварительными знаниями в этих областях.

\section*{Благодарности}

Для нас очень важными были помощь и комментарии Хью Глэйзера (Hugh Glaser), Энди Грэйвелла (Andy Gravell), Лауры Лафев (Laura Lafave), Денис Николь (Denis Nicole), Питера Сестофта (Peter Sestoft) и анонимных читателей. Материал книги прошёл первое испытание в Саутгемптоне в 1995/1996 годах. Большое количество полезных отзывов пришло от студентов первого курса 1995 года набора, в частности от Джейсона Датта (Jason Datt) и Алекса Уолкера (Alex Walker).

В процессе работы над книгой мы использовали множество общественно доступных программных инструментов. Особенно полезными были: система грамотного программирования \lstinline|noweb| Нормана Рамси ({\selectlanguage{english}Norman Ramsey}), инструменты построения синтаксических диаграмм от L. Rooijakkers, \lstinline|gpic| от Брайана Кернигана (Brian Kernighan), \TeX, \LaTeX, New Jersey SML, и компилятор Gnu C.

Gnu C-compiler\texttrademark --- торговая марка Free Software Foundation.

IBM PC\texttrademark --- торговая марка IBM.

Macintosh\texttrademark --- торговая марка Apple Computer, Inc.

Miranda\texttrademark --- торговая марка Research Software Ltd.

UNIX\texttrademark --- торговая марка Novell.

Postscript\texttrademark --- торговая марка Adobe Systems, Inc.

\TeX\texttrademark --- торговая марка American Mathematical Society.

X Window System\texttrademark --- торговая марка MIT.



\chapter{Введение}
Программирование --- это деятельность, заключающаяся в инструктировании компьютера с тем, чтобы он помогал решать задачи. Эти инструкции могут быть подготовлены на основе некоторого числа парадигм. Эта книга написана для тех, кто знаком с функциональной парадигмой, использующих SML как язык программирования и для тех, кто желает изучить программирование в императивной парадигме используя C как язык программирования.

\section{Функциональная и императивная парадигмы}

Функциональная и императивная парадигмы действуют с различных точек зрения. Функциональная парадигма основывается на \emph{вычислении выражений} и привязке переменных к значениям. Базовая программная фраза --- это выражение; цель вычисления выражения заключается в получении значения. Порядок в котором вычисляются подвыражения не влияет на результирующие значения.

Императивная парадигма основывается на \emph{исполнении операторов} и наличии хранилища, в котором операторы могут оставлять их результаты работы. Базовая программная фраза --- это оператор; цель заключающаяся в исполнении оператора --- это изменение хранилища. Порядок, в котором выполняются операторы влияет на результирующее значение хранилища. Текущий набор значений в хранилище называется \emph{состоянием} программы.

Причины возникновения этих различных подходов лежат в происхождении языков, основывающихся на этих парадигмах. Функциональные языки были разработаны из математики. Основа этих языков --- чистота и лёгкость в понимании. Императивные языки были созданы исходя с машинной точки зрения: архитектура фон Неймана обладает памятью и процессором, оперирующим в этой памяти. Императивная парадигма --- это высокий уровень абстракции этой модели.

Функциональные и императивные языки могут быть использованы для программирования в обоих стилях: возможно писать императивные программы в SML и также возможно писать функциональные программы в C. Однако такие программы часто 'неестественны', в том смысле, что их формулировка неуклюжа поскольку язык не предполагает наличия наиболее подходящих абстракций.

В качестве примера рассмотрим классическую задачу генерации псевдо случайных чисел. Эту проблему мы можем функционально сформулировать в C таким образом:

\begin{lstlisting}
int functional_random( int seed ) {
  return 22 * seed % 37 ;
}
\end{lstlisting}

Функциональную реализацию следует читать таким образом: первая строка вводит функцию под названием \lstinline|functional_random|. Эта функция принимает целочисленный аргумент названный как \lstinline|seed| и возвращает также целочисленное значение. Тело функции (окружённое фигурными скобками \lstinline|{| и \lstinline|}|) содержит единственный оператор возврата. Аргумент оператора возврата представляет выражение, умножающее значение \lstinline|seed| на \lstinline|22| и возвращающее остаток от деления полученного результата на \lstinline|37|.

Для использования функции \lstinline|functional_random|, следует выбрать подходящее стартовое значение для seed, например 1, и затем циклично применять функцию \lstinline|functional_random|. Таким образом можно получить последовательные псевдо случайные значения. Рассмотрим следующий фрагмент программы на C:

\begin{lstlisting}
int first = functional_random( 1 ) ;
int second = functional_random( first ) ;
int third = functional_random( second ) ;
\end{lstlisting}

Значением переменной first будет 22; значением переменной second будет 3, (поскольку 22*22 = 13*37 + 3) и значение third будет равно 29.

Функция \lstinline|functional_random| - чистая функция, это значит, что значение возвращаемое каждым вызовом \lstinline|functional_random| зависит исключительно от значения её аргумента. В свою очередь это означает, что \lstinline|fucntional_random| будет всегда возвращать один и тот же ответ при её вызове с одним и тем же значением аргумента. Следовательно, чистая функция - это хороший строительный блок.

Также генератор псевдо случайных чисел может быть написан в императивном стиле:

\begin{lstlisting}
int seed = 1 ;
int imperative_random( void ) {
seed = 22 * seed % 37 ;
return seed ;
}
\end{lstlisting}

Императивную реализацию следует читать так: первая строка определяет глобальную переменную, названную как seed, хранящее начальное значение. Начальное значение равно 1. Следующая строка вводит функцию под названием \lstinline|imperative_random|. Функция не имеет аргументов, что обозначается словом void. Функция изменяет значение seed и возвращает это значение после изменения. Модификация хранилища являет собой побочный эффект, поскольку этот эффект был дополнительным к возвращению псевдо случайного числа.

То, что эта функция императивна становится ясным после того, как мы исполним этот код 'в уме'. Первый вызов \lstinline|imperative_random| вернёт 22, после чего переменная seed получит значение 22. Это приведёт к тому, что функция вернёт 3 на следующем вызове. Таким образом, функция \lstinline|imperative_random| будет возвращать новое значение, а это именно то, что требовалось нам от генератора псевдо случайных чисел. Порядок вызовов становится важным, поскольку теперь значение, возвращаемое от \lstinline|imperative_random| зависит от состояния, а не от её аргумента.

У обеих парадигм есть свои преимущества. Императивная парадигма облегчает работу с состоянием, а состояние не должно передаваться от одной функции к другой, состояние всегда существует. Функциональная парадигма позволяет нам создавать строительные блоки, которые можно использовать более свободно. На этих вопросах мы остановимся позже.

1.1.1 Преимущество состояния

Полезным расширением случайной функции будет создание функции возвращающей значение игральной кости. Значением игральной кости будет получение остатка от деления случайного числа на 6, добавление к полученному числу единицы, что даст значение в промежутке 1...6. Императивная функция для игральной кости будет такой:

\begin{lstlisting}
int imperative_dice( void ) {
return imperative_random() % 6 + 1 ;
}
\end{lstlisting}

Случайное число создано, выполнена операция получения остатка от деления на 6 и добавлена единица. Создание функциональной версии будет более тяжёлым, требуется возвращать два числа от функции: значение игральной кости и состояние генератора случайных чисел. Вызывающая сторона функциональной игральной кости должна получить одно число и запомнить другое для использования в следующем вызове.

1.1.2 Преимущества чистых функций

Хранение состояния в каком либо скрытом месте приводит к неудобству: усложняется создание функций, используемых в качестве строительных блоков. В качестве примера предположим, что нам нужно одновременно бросать две игральные кости. Теория генераторов случайных чисел говорит, что будет некорректным использовать одно число от одного генератора случайных чисел для генерации значений двух игральных костей [5]. Вместо этого следует использовать два независимых генератора случайных чисел.

Функциональная версия предлагает использовать в качестве строительного блока генератор случайных чисел, предполагая, что инициализирующие значения были сохранены в r и s, тогда следующий фрагмент кода будет генерировать значение для броска двух костей:

int x = functional_random( r ) ;
int y = functional_random( s ) ;
int dice = x%6 + 1 + y%6 + 1 ;

Этого невозможно получить с помощью императивной версии, поскольку есть только одна переменная seed хранящая начальное значение.

1.1.3 Построение блоков характерное для C

В идеале нам нужно обладать лучшим из обоих миров. Читатель может заглянуть в Главу 8 - в этой главе дан генератор случайных чисел, являющийся хорошим строительным блоком и передающий состояние так, что результат хорошо масштабируется за пределы одной функции.

Цель книги можно выразить так: мы хотим создать высоко эффективный и характерный для C код, но, вместе с тем хотим спроектировать хорошие строительные блоки, сохранив все техники, являющиеся общим знанием в мире функционального программирования. Примеры включают чистые функции, полиморфные функции, каррирование, алгебраические типы данных и рекурсию.

1.2 Руководство по книге

В следующей главе обсуждается базовая исполнительная модель C. Также представлены базовый синтаксис и типы данных. В этой главе используется декларативное подмножество C. Оно не позволяет писать эффективный код в C стиле, но служит для знакомства читателя с базовым синтаксисом C. Кроме того во второй главе вводится первая систематичная трансформация функционального кода в C код.

В Главе 3 обсуждается итерация. Функциональные языки осуществляют итерацию через структуры данных с помощью прямой рекурсии или косвенной рекурсии посредством таких высокоуровненвых функций, как map и foldl. C предлагает конструкции, которые выполняют итерацию без рекурсии с помощью циклично исполняемых определённых частей программного кода. В Главе 3 мы создаём эффективный код в C стиле для большинства примеров программ Главы 2. Это будет выполнено с помощью определённого числа систематических, но неформальных схем трансформации программ.

В Главе 4 обсуждаются конструкторы типов C, необходимых для создания (не рекурсивных) алгебраических типов данных. Эти конструкторы называются структурами или объединениями. В главе обсуждается вопрос создания составных типов и заканчивается темой деструктивных обновлений этих структур данных (с помощью указателей).

В первых четырёх главах обсуждаются базовые типы данных и их C представления. Более сложные типы данных могут хранить последовательности данных. Для хранения последовательностей есть некоторое число представлений. В функциональных языках популярны списки; массивы используются для эффективного доступа к случайным элементам. Списки менее популярны в C поскольку управление списками в C более трудоёмко чем управление списками в функциональных языках. Последовательности, массивы, списки и потоки обсуждаются в Главах 5 и 7.

Глава 5 представляет базовые принципы последовательностей и реализацию массивов. Массивы в C являются низкоуровневыми, но они показаны как создаваемые высокоуровневые структуры в функциональных языках. Списки обсуждаются в Главе 6. Реализация списков требует явного управления памятью; это одна из причин. по которой использование списков менее удобно в C чем в SML. Поток, список элементов используемых или записываемых последовательно (как Ввод/Вывод (I/O)), является темой Главы 7.

В Главе 8 в деталях рассматривается работа модульной системы C и выполняется её сравнение с модульной системой SML. Модульное программирование - ключевая проблема в программной инженерии. В 8-й главе затрагиваются вопросы правильного использования состояния и определения интерфейсов позволяющих реализовать чистые функции в модулях.

В Главе 9, последней главе, показаны три обучающих примера элементарной графики. Первый пример полностью проработан, он показывает как использовать систему X-window для рисования фрактала для множества Мандельброта. Второй обучающий пример проработан частично, он описывает систему для платформно независимой графики. Большую часть примера предстоит реализовать читателю. В третьем примере разрабатывается реализация простого графического языка. Описан алгоритм и даны наброски структур данных, реализация этого примера оставлена читателю.

Приложение A содержит ответы на некоторые упражнения. Упражнения дают читателям возможность проверить и улучшить свои навыки. Есть два типа упражнений. Простые упражнения улучшают общие навыки решения задач. Многие из них требуют реализации некоторого SML кода и почти все из них требуют реализации некоторых C функций. Упражнения помеченные астериском предназначены для читателей, заинтересованных в фундаментальных вопросах программирования. Все доказательства теорем оставленные для читателя помечены как 'упражнение*'.

Приложение B - это краткий обзор SML для людей, знакомых с другими функциональными языками. Предполагается, что человек может читать SML программы. Мы только обсуждаем подмножество SML используемое в этой книге и только в терминах других функциональных языков программирования. Также в этом приложении обсуждаются применяемые нами (маленькое множество) SML-овские библиотечные функции.

В Приложении C перечислены библиотечные функции C. Все языки программирования поставляются с набором примитивных операторов и типов данных в виде коллекции библиотечных модулей предназначенных для большего удобства. C не исключение и обладает большим набором библиотек. Мы приводим небольшое количество функций, присутствующих в библиотеке C.

Последнее Приложение D даёт полный синтаксис ISO-C с помощью синтаксических диаграмм. Это интуитивно понятней чем альтернативная БНФ нотация для синтаксиса. Синтаксическая диаграмма представлена как справочное пособие.

Эта книга не является полноценным справочным материалом. Читатель найдёт полезным обратиться к справочному материалу ISO-C [7]. Он содержит все детали, которые могут понадобится опытному C программисту, а также вводный материал не упомянутый в этой книге.

Переведено на Нотабеноиде
http://notabenoid.org/book/60709/295443

Переводчики: Charlz_Klug

%\input{Chapters/running_examples}
%\part{Введение}
%\input{Chapters/01_Chapter}
%\part{Общие Модели Вычислений}
%\input{Chapters/02_Chapter}



%\part{Приложения}
%\begin{thebibliography}{17}
  \addcontentsline{toc}{chapter}{Литература}
  \selectlanguage{english}

\bibitem{1}
  A. V. Aho, R. Sethi, and J. D. Ullman. \emph{Compilers: Principles, techniques, and tools.} Addison Wesley, Reading, Massachusetts, 1986.

\bibitem{2}
  J. L. Hennessy and D. A. Patterson. \emph{Computer architecture: A quantitative approach.} Morgan Kaufmann Publishers, Inc., San Mateo, California, 1990.

\bibitem{3}
  C. A. R. Hoare. Algorithm 64 quicksort. \emph{CACM}, 4(7):321, Jul 1961.

\bibitem{4}
  Adobe Systems Inc. \emph{PostScript language reference manual.} Addison Wesley, Reading, Massachusetts, 1985.

\bibitem{5}
  R. Jain. \emph{The art of Computer Systems Performance Analysis.}  John Wiley, Newyork, 1991.

\bibitem{6}
  B. W. Kernighan. PIC --- a language for typesetting graphics. \emph{Software~---~practice and experience,} 12(1):1–21, Jan 1982.

\bibitem{7}
  B. W. Kernighan and D. W. Ritchie. \emph{The C programming language~---~ANSI C.} Prentice Hall, Englewood Cliffs, New Jersey, second edition edition, 1988.

\bibitem{8}
  D. E. Knuth. \emph{The art of computer programming, volume 1: Fundamental algorithms.} Addison Wesley, Reading, Massachusetts, second edition, 1973.

\bibitem{9}
  L. C. Paulson. \emph{ML for the working programmer.} Cambridge Univ. Press, New York, 1991.

\bibitem{10}
  W. H. Press, B. P. Flannery, S. A. Tekolsky, and W. T. Vetterling. \emph{Numerical recipes in C --- The art of scientific computing.} Cambridge Univ. Press, Cambridge, England, 1993.

\bibitem{11}
  B. Schneier. \emph{Applied cryptography.} John Wiley \& Sons, Chichester, England, second edition edition, 1996.

\bibitem{12}
  R. Sedgewick. \emph{Algorithms.} Addison Wesley, Reading, Massachusetts, 1983.

\bibitem{13}
  E. H. Spafford. The internet worm program: an analysis. \emph{ACM Computer communication review,} 19(1):17–??, Jan 1989.

\bibitem{14}
  A. S. Tanenbaum. \emph{Structured computer organisation.} Prentice Hall, Englewood Cliffs, New Jersey, second edition, 1984.

\bibitem{15}
  J. D. Ullman. \emph{Elements of ML programming.} Prentice Hall, Englewood Cliffs, New Jersey, 1994.

\bibitem{16}
  A. Wikstr\"om. \emph{Functional programming using Standard ML.} Prentice Hall, London, England, 1987.

\bibitem{17}
  X-Consortium. \emph{X Window Manuals.} O’Reilly \& Associates, Inc., New York, 1990.

\end{thebibliography}

\end{document}
